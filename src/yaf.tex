\part{Yaf}


\chapter{Overview}

Yaf(Yet Another Framework)扩展是一个用来开发web应用的php框架,构建Yaf扩展不需要其他扩展,而且Yaf并未与PHP进行捆绑,需要开发者自己单独安装。


\section{Directory}


一个典型的Yaf应用程序的目录结构如下:

\begin{lstlisting}[language=bash]
|— index.php
|— .htaccess
|— conf
|       `— application.ini // application config
|— application
|       |— Bootstrap.php
|       |— controllers
|       |       |— Index.php // default controller
|       |— views
|       |       |— index.phtml // view template for default action
|       `— modules
|— library
|— models
|— plugins       
\end{lstlisting}

\section{Intry Script}


Yaf应用程序根目录下的index.php是整个Yaf应用程序唯一的入口,应该把所有请求都重定向到这个文件(在Apache+php\_mod模式下可以使用.htaccess)。



\begin{lstlisting}[language=PHP]
<?php
define("APPLICATION_PATH",dirname(__FILE__));

$app = new Yaf_Application(APPLICATION_PATH . "/conf/application.ini");
$app->bootstrap() // call bootstrap methods defined in Bootstrap.php
     ->run();
?>
\end{lstlisting}

\section{Rewrite Rule}


Yaf使用的URL重写规则如下:


\begin{lstlisting}[language=bash]
# apache(.htaccess)
RewriteEngine On
RewriteCond %{REQUEST_FILENAME} !-f
RewriteRule .* index.php

# for nginx
server {
   listen 80;
   server_name _;
   root document_root;
   index index.php index.html index.htm;
   
   if ( !-e $request_filename ) {
      rewrite ^/(.*) /index.php/$1 last;
   }
}

# for lighttpd
$HTTP["host"] =~ "(www.)?domain.com$" {
   url.rewrite = (
       "^/(.+)/?$" => "/index.php/$1",
   )
}
\end{lstlisting}

另外,在LNMP架构中使用Nginx和PHP-FPM时,可以使用如下的URL重写配置:

\begin{lstlisting}[language=bash]
server {
   listen 80;
   server_name _;
   root document_root;
   index index.php index.html index.htm;
   
   location / {
      try_files $uri $uri/ /index.php$is_args$args;
   }
   
   location ~ \.php$ {
       fastcgi_pass 127.0.0.1:9000;
       #fastcgi_pass unix:/tmp/php-cgi.sock;
       fastcgi_index index.php;
       fastcgi_param SCRIPT_FILENAME $document_root$fastcgi_script_name;
       include fastcgi_params;  
   }
}
\end{lstlisting}


\subsection{PATH\_INFO}

在使用Nginx作为单一入口的PHP框架的反向代理时,需要使用Nginx向Yaf/ThinkPHP等应用程序传递PATH\_INFO请求参数才能正常运行,例如:


\begin{lstlisting}[language=bash]
server {
   listen 80;
   server_name _;
   root document_root;
   index index.php index.html index.htm;
   
   location / {
      try_files $uri $uri/ @path_rw;
   }
   
   location @path_rw {
      rewrite ^/(.*)$ /index.php/$1 last;
   }
   
   location ~ \.php {
       fastcgi_pass 127.0.0.1:9000;
       #fastcgi_pass unix:/tmp/php-cgi.sock;
       fastcgi_index index.php;
       include fastcgi.conf;
       
       fastcgi_split_path_info ^(.+?\.php)(/.+)$;
       fastcgi_param SCRIPT_FILENAME $document_root$fastcgi_script_name;
       fastcgi_param SCRIPT_NAME $fastcgi_script_name;
       fastcgi_param PATH_INFO $fastcgi_path_info;
       #include fastcgi_params;  
   }
}
\end{lstlisting}

\section{Configuration}



Yaf默认的应用配置文件application.ini的内容如下:

\begin{lstlisting}[language=bash]
[yaf]
;APPLICATION_PATH is the constant defined in index.php
application.directory = APPLICATION_PATH "/application"

;product section inherit from yaf section
[product:yaf]
foo=bar
\end{lstlisting}

Yaf默认的Bootstrap.php的内容如下:

\begin{lstlisting}[language=PHP]
<?php
/* Bootstrap class should be define under ./application/Boostrap.php */
class Boostrap extends Yaf_Bootstrap_Abstract {
   public function _initConfig(Yaf_Dispatcher $dispatcher) {
      var_dump(__METHOD__);
   }
   
   public function _initPlugin(Yaf_Dispatcher $dispatcher) {
      var_dump(__MATHOD__);
   }
}
\end{lstlisting}


\section{Controller}


Yaf默认的控制器为controllers/Index.php,其内容如下:

\begin{lstlisting}[language=PHP]
<?php
class IndexController extends Yaf_Controller_Abstract {
    /* default action */
    public function indexAction() {
        $this->_view->word = "hello world, yaf";
        // or
        // $this->getView()->word = "hello world, yaf";
    }
}
\end{lstlisting}


\section{View}


Yaf默认的视图文件为views/index.phtml,其内容如下:

\begin{lstlisting}[language=HTML]
<html>
  <head>
    <title>Hello world</title>
  </head>
  <body>
     <?php echo $word; ?>
  </body>
</html>
\end{lstlisting}

运行上述的Yaf示例应用程序的输出类似于:


\begin{lstlisting}[language=bash]
<html>
  <head>
    <title>Hello world</title>
  </head>
  <body>
     hello world, yaf
  </body>
</html>
\end{lstlisting}




\begin{lstlisting}[language=bash]

\end{lstlisting}